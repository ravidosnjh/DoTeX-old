%               DEPROCLDC.TeX
%
%  This file is the preliminary version of a talk given at the Spring '86
%  DECUS in Dallas by Barbara Beeton, American Mathematical Society.
%  It describes the method by which an author of an article for the DECUS
%  Proceedings can prepare his camera copy using LaTeX.
%
%  This file, when run through LaTeX, will be formatted as described below.
%  It uses the macro package DEPROC.STY and the LaTeX ARTICLE.STY files,
%  plus a few locally defined macros, which are given at the top of this file.
%
%  One page in the output will be blank, reserved for a table.  The
%  contents of the table are in the file DEPROCDEV.TeX, which is to be
%  run through TeX as a separate job.
%
%  Although essentially complete, and to the best of my knowledge correct,
%  some details can be expected to change prior to publication.
%
%  Barbara Beeton
%  American Mathematical Society
%  P O Box 6248
%  Providence, RI 02940
%    (401)272-9500
%
%  20 Apr 86

%  If TeX is the program being run, this \input statement is required.
%\input latex
\documentstyle [draft]{deproc}

%       try to force a blank page in the middle for the floating figure

\def\dblfloatpagefraction{.5}
\def\textfraction{0}
\def\floatpagefraction{.5}

%       macros needed for this article

\newcommand{\AMS}{American Mathematical Society}
\newcommand{\Proc}{Proceedings}
\newcommand{\DP}{\mbox{\tt DEPROC}}
\newcommand{\DProc}{{\sl DECUS \Proc}}
\newcommand{\POBox}{P.\thinspace O.~Box }
\newcommand{\TB}{{\sl The \TeX book}}
\newcommand{\tub}{TUGboat}
\newcommand{\TUG}{\TeX\ Users Group}
\newcommand{\VAX}{\leavevmode\hbox{V\kern-.12em A\kern-.1em X}}

%\def{\<#1>{$\langle${#1}$\rangle$}
%\newcommand{\cs}{\<cs>}
%\newcommand{\css}{\cs es}
\newcommand{\cs}{{\tt cs}}
\newcommand{\css}{\cs-es}
\newcommand{\CR}{$\langle${\sc cr}$\rangle$}
\newcommand{\tab}{$\langle${\sc tab}$\rangle$}
\newcommand{\ttvert}{{\tt\char'174 }\ignorespaces}
\newcommand{\allowhbreak}{\penalty0\hskip0pt}

\makeatletter
\newlength{\n@rmal@labelwidth}\setlength{\n@rmal@labelwidth}{\labelwidth}
\newlength{\n@rmal@labelsep}\setlength{\n@rmal@labelsep}{\labelsep}
\def\pseudobibliography#1{\setbox0=\hbox{#1}\list
  {\ifomit@biblabels \else [\arabic{enumi}]\fi}{\ifomit@biblabels
  \setlength{\labelwidth}{0pt}\setlength{\labelsep}{0pt}\else
  \settowidth\labelwidth{[#1]}\fi\leftmargin\labelwidth
  \advance\leftmargin\labelsep
  \ifomit@biblabels\else\usecounter{enumi}\fi}}
\def\endpseudobibliography{\endlist \global\omit@biblabelsfalse
  \setlength{\labelwidth}{\n@rmal@labelwidth}%
  \setlength{\labelsep}{\n@rmal@labelsep}}
\makeatother
\def\pseudocite#1{[#1]}










\begin{document}

\title{Typesetting Articles for the DECUS
    Proceedings with \LaTeX}
\author{Barbara N. Beeton\\
    \AMS\\
    Providence, Rhode Island}

\begin{abstract}
The \DProc\/ have traditionally been published from copy supplied by
the authors, prepared according to rules devised for typewritten
material.  The power of the computer typesetting language
\TeX,\footnotemark\ through the macro package \LaTeX, has
now been applied to this task, and a formatting package, named \DP,
has been submitted to the DECUS Program Library for use by authors who
have access to a working \TeX\ system.  (The \TeX\ program and related
software, created by Donald Knuth of Stanford, are in the public domain.)

This paper presents the important features of the \LaTeX\ implementation
of \DP\ and, through
examples, shows how it is to be used.  Use of \DP, which is
encouraged, will produce the author's work, nicely typeset, in the
standard {\sl\Proc\/} format.  There is a general description
of how the package works and of the mechanical requirements for camera
copy of {\sl\Proc\/} articles, which will be created on the
author's local output device.

No prior knowledge of \TeX\ or \LaTeX\ is required, but authors using \DP\ will
be expected to learn some rudiments, especially if their papers
contain special notation or formats such as tables.
\end{abstract}

%\begin{document}

\maketitle

\footnotetext[1]{\TeX\ is a trademark of the \AMS.}

The \DProc, like the conference proceedings of many other organizations,
is rushed to publication as quickly as possible so that the material
will reach the conference participants and other interested readers
before its value is diminished by time.  Reproducing author-prepared
copy eliminates the considerable bother and expense of typesetting,
proofreading and corrections.  The published document should be
compact, uniform in appearance, and readable, regardless of the kind
or quality of printing device available to the author.  For these
reasons, instructions to authors have heretofore assumed that nothing
more elaborate is available than an ordinary typewriter or dot matrix
printer.

To enforce uniformity, the author is provided with ``model paper'', on
which are printed (in non-reproducing ink) column and page borders,
alignment marks, and instructions for placement of title, author, and
the other parts of a proceedings article.  The dimensions of the model
paper are almost always larger than those of the published \Proc\Dash
this permits more text to be packed onto each page, and also improves
its appearance or ``quality'' when photographically reduced, smoothing
out the rough edges of letters and symbols generated by a typewriter,
dot-matrix printer or other ``low-resolution'' device.

Within the past few years, advances in laser-printer technology have
made good-quality output accessible to a growing number of users,
through a widening selection of low-cost output systems based on print
engines with 300 dot-per-inch resolution and (relatively) easy-to-use
interfaces.  Such devices have been attached to most kinds of DEC
computers, and drivers now exist to print the output from such programs
as Scribe,\footnote{Scribe is a trademark of Unilogic Ltd.}
\TeX\ and Troff.  Most low-end laser printers cannot use paper wider
than $8\sfrac1/2''$, however, so even if both a good composition
program and output printer had been available, until now an author
would have been discouraged from using them for mechanical reasons.

The editor of the \DProc\/ has now agreed to accept typeset copy
printed on such a system at 100\% on $8\sfrac1/2\times11''$ paper,
provided it conforms to the published format.  This article (which has
itself been produced by the technique it describes) introduces a
package, \DP, designed to prepare {\sl\Proc\/} articles using \LaTeX.











\newcommand{\Dag}{$^{\thinspace\dagger}$}
\newcommand{\1}[1]{{\let\\=\newline\parbox[t]{.177\hsize}{\raggedright#1}}}
\newcommand{\8}{\vrule height .9\baselineskip depth 0pt width 0pt\ignorespaces}
\newcommand{\9}{\vrule height 0pt depth .4\baselineskip width 0pt}
\newcommand{\KnS}{Kellerman\thinspace\&\thinspace Smith}
\begin{figure*}[p]
\begin{tabular}{|l|l|l|l|l|}                    \hline
\1{\8\9}& \multicolumn{1}{c}{DECSystem-10}
                & \multicolumn{1}{c}{DECSYSTEM-20}
                        & \multicolumn{1}{c}{VAX (Unix)}
                                & \multicolumn{1}{c}{VAX (VMS)} \\ \hline
\1{\8 Allied Linotronic\\L100, L300P\9}
        &       &       & \1{Textset}
                                & \1{Textset} \\ \hline
\1{\8 Apple LaserWriter\\ \9}
        &       &       & \1{Carleton University\\Textset\Dag}
                                & \1{Textset\Dag} \\ \hline
\1{\8 Autologic\\APS-5, Micro-5\9}
        &       & \1{Textset}
                        & \1{Textset}
                                & \1{Intergraph\Dag\\Textset} \\ \hline
\1{\8 Canon\\ \9}
        &       &       & \1{Canon} &   \\ \hline
\1{\8 Compugraphic\\8400, 8600\9}
        &       &       &       & \1{\KnS}      \\ \hline
\1{\8 DEC LN01\\ \9}
        &       &       & \1{Univ of Washington}
                                & \1{Louisiana State U} \\ \hline
\1{\8 DEC LN03\\ \9}
        &       &       &       & \1{DEC\\ \KnS} \\ \hline
\1{\8 Imagen\\ \9}
        & \1{Stanford\\Vanderbilt}
                & \1{SRI\\Columbia}
                        & \1{Univ of Maryland}
                                & \1{\KnS\Dag} \\ \hline
\1{\8 QMS Lasergrafix\\ \9}
        &       &       & \1{Textset\\Univ of Washington}
                        & \1{GA Technologies\\Texas A\&M\\Textset\9} \\ \hline
\1{\8 Symbolics\\ \9}
        &       & \1{Univ of Washington}
                        & \1{Univ of Washington}
                                & \1{UMass} \\ \hline
\1{\8 Talaris\\ \9}
        & \1{Talaris\Dag}
                & \1{Talaris\Dag}
                        & \1{Talaris\Dag}
                                & \1{Talaris\Dag} \\ \hline
\1{\8 Xerox Dover\\ \9}
        &       & \1{Carnegie-Mellon U}
                        & \1{Stanford}  & \\ \hline
\1{\8 Xerox 2700\\ \9}
        &       & \1{Ohio State U}
                        & \1{Ohio State U}      & \\ \hline
\1{\8 Xerox 9700\\ \9}
        & \1{Univ of Delaware}
                &       & \1{Univ of Delaware}
                                & \1{ACC\\Textset} \\ \hline
\end{tabular}
\par\vspace{2pt}
\mbox{\Dag\thinspace Graphics supported}
\vspace{-.5\baselineskip}
\caption[outdev]{Computer/output device combinations with \TeX\ interfaces}
\vspace{.5\baselineskip}
\label{outdev}
\end{figure*}
\newcommand{\site}[1]{\par \noindent \hangindent 20pt
                {\bf #1}\quad\ignorespaces}
\begin{table*}[p]
\hbox to\textwidth{%
\parbox[t]{.47\textwidth}{%
  Information regarding the interfaces shown here can be obtained
  from the individual listed below for the site.  This table and
  the names of the site contacts were provided by the \TeX\ Users Group.

\raggedright \hyphenpenalty=10000 \exhyphenpenalty=10000
\medskip
\site{ACC {\rm(Advanced Computer Communications)}} Diane~Cast, 805-963-9431

\site{Canon {\rm (Tokyo)}} Masaaki Nagashima, (03) 758-2111

\site{Carleton University} Neil Holtz, 613-231-7145

\site{Carnegie-Mellon University} Howard Gayle, 412-578-3042

\site{Columbia University} Frank da Cruz, 212-280-5126

\site{DEC {\rm(Digital Equipment Corp)}} John Sauter, 603-881-2301

\site{GA Technologies} Phil Andrews, 619-455-4583

\site{Intergraph} Mike Cunningham, 205-772-2000

\site{\KnS} Barry Smith, 503-222-4234

}%              end \parbox
\hfill
\parbox[t]{.47\textwidth}{%
\raggedright \hyphenpenalty=10000 \exhyphenpenalty=10000
\site{Louisiana State University} Neil Stoltzfus, 504-388-1570

\site{Ohio State University} John Gourlay, 614-422-6653

\site{SRI}

\site{Stanford}

\site{Talaris} Sonny Burkett, 619-587-0787

\site{Texas A\&M} Bart Childs, 409-845-5470

\site{Textset} Bruce Baker, 313-996-3566

\site{University of Delaware} Daniel Grim, 302-451-1990

\site{University of Maryland} Chris Torek, 301-454-7690

\site{University of Massachusetts} Gary Wallace, 413-545-4296

\site{University of Washington} Pierre MacKay, 206-543-2386

\site{Vanderbilt University} H. Denson Burnum, 615-322-2357

}%              end \parbox
}%              end \line
\end{table*}











\section{What is \TeX? What is \LaTeX?}

\TeX\ is a public-domain typesetting language created by Donald Knuth
of Stanford University.  His original aim was to typeset his own books,
in particular {\sl The Art of Computer Programming\/} \cite{ACP}, with
a quality equal to that produced by the best traditional composition
methods.  The technical content of these books assured that full
attention was given to the niceties of formatting mathematical
expressions, as well as to the structures of documents commonly
encountered in technical publishing.

\TeX\ deals with low-level concepts familiar to typesetters\Dash type
size, leading, interword spacing and kerning.  It does not incorporate
directly the structures an author encounters when writing a paper\Dash
title, figure references, bibliographic entries.  However, \TeX\ is
essentially a macro compiler, and provides a full vocabulary of low-level
functions that can be manipulated by knowledgeable users to create
higher-level packages to support the casual user.

One such macro package is \LaTeX.  \LaTeX\ \cite{LT} is
a powerful document formatter, providing the capability to format
books and reports, with functionality similar to that provided by
Scribe.











\section{The \DP\ macro package}

In order to use this implementation of the \DP\ macro package, the
author of a \DProc\/ article must have available a working \LaTeX\ system,
which presupposes a working \TeX\ system.
\TeX\ has been implemented on \VAX es and DECsystem-10s and -20s
under the standard operating systems.
There is also a good selection of output devices available, capable
of production output of quality suitable for the {\sl\Proc\/};
Table~\ref{outdev}
shows the computer/output device combinations known to the \TUG.
(\TeX\ has not, however, been implemented on PDP-11s, since
it requires a larger address space than is supported on those machines.)

\LaTeX\ may not be available to all \TeX\ users.  (\TeX\ is a very large
program by itself, and routinely adding a large macro package can put
unwelcome strain on an already overloaded machine.  Some system
administrators prefer not to give their users that opportunity.)
An earlier implementation of \DP\ does not require \LaTeX, but only
\TeX\ itself; it was
described in \cite{DP}, and the supporting files are on the Fall~'85
DECUS Program Library tapes for Languages \& Tools, Large Systems, and \VAX.

The present \LaTeX-based version of this macro package
is called \verb|DEPROC.STY|, for ``\DProc\/ style file''.  It is
an ordinary ASCII file, and has been submitted to the Spring~86 DECUS
Program Library for the same systems listed above.











\section{Some preliminary \TeX nical information}

An author who intends to use the \LaTeX\ version of \DP\ should
preferably have used \LaTeX\ already.  Nonetheless, a few basic
concepts are worth repeating.  (\LaTeX\ is identical to \TeX\ in
many ways.  The following discussion will specify \LaTeX\ only when
there is a difference.)

\subsection{Spacing}

\TeX\ uses different spacing rules in text (paragraphs) and math.
Paragraphs are set so that interword spacing is as uniform as possible.
Wider spaces are set after punctuation that indicates the ends of
sentences (period, !\ and ?).  Within math, the best traditions for
arranging symbols in two dimensions, including proper spacing, are
observed.  Thus input spacing is largely ignored, except
for its functions of separating words and marking the boundaries of
certain kinds of expressions.  \TeX\ considers
multiple spaces in an input file to be equivalent to a single space.
The carriage return \CR\ and the tab character \tab\ are equivalent
to ordinary spaces, except in special environments (noted below).
And all spaces at the beginning of any line are ignored.

\subsection{Paragraph breaks}

A blank line in the input file indicates a paragraph break.  A line is
blank if it contains only a \CR\ or spaces and a \CR.  Multiple blank
lines are equivalent to a single blank line.  (A paragraph can also be
indicated by \verb|\par|; terms beginning with \verb|\| are described below.)

\subsection{Comments}

A comment may be entered on any line; a comment begins with a \verb|%| sign:
\begin{verbatim}
%  This line contains nothing but a comment.
\newcommand{\cs}{...}%  explanatory comment
... Smythe % ***** check spelling *****
\end{verbatim}
\TeX\ will ignore the \verb|%| and everything following it, including the \CR.
Thus, the space ordinarily indicated by the \CR\ will be suppressed,
and if a space is really wanted between the last item before a comment
and the first item on the next line, it must be input before the \verb|%|.
Conversely, if no space is wanted between the last item on a line and
the first item on the next, a \verb|%| can be used to suppress it
intentionally.

\subsection{Control sequences, also called macros}

A ``control sequence'' \cs\ is an instruction for \TeX\ to perform
some action or to produce a particular symbol.  A \cs\ begins with a
backslash, \verb|\|.  There are two types of \css:
\begin{itemize}
\item[--] A ``control word'' consists of \verb|\| followed by one or more
        letters.  It is terminated by any non-letter, including a space;
        multiple spaces follow the usual compression rule, so a
        special technique (see next paragraph) is required to create an
        output space after a control word.  \verb|\TeX| is an example of
        a control word; it produces the \TeX\ logo.
\item[--] A ``control symbol'' consists of \verb|\| followed by exactly one
        non-letter.  Since its length is known, no special terminator
        is required.  \verb|\&| is a control symbol to produce an \&.
        \verb*|\ | (\verb|\| followed by a space) is an explicit space, to
        be used where an output space should follow an element
        input as a control word.
\end{itemize}

\noindent
New \css\ can be defined within a document to make input easier or clearer.
A few principles governing \cs\ names should be observed carefully.
\begin{itemize}
\item[--] Case matters; \verb|\csname| is not the same as
        \verb|\Csname| or \verb|\CSName|.
        Try to pick a name that means something to you, and is easy to type.
\item[--] Don't try to redefine an existing \cs\ name unless you really
        know what you're doing; results, as they say, ``may be unpredictable''.
\item[--] Never define or redefine any \cs\ whose name begins with
        `\verb|\end|'.
\end{itemize}

\noindent
To define a new command,
\begin{verbatim}
\newcommand{\csname}{...something...}
\end{verbatim}
If the name has been used before, \LaTeX\ will stop and report an
error.  If you are really adamant about re-using this name,
\begin{verbatim}
\renewcommand{\csname}{...something...}
\end{verbatim}
will assign it the new meaning.

The control symbols \verb|\0,...,\9| always start out undefined, so they are
available for transient use without checking.

A \cs\ with arguments is defined by
\begin{verbatim}
\newcommand{\csname}[2]{...#1...{#2}...}
\end{verbatim}
with the number of arguments given in brackets as shown; for details,
see \cite{LT}.

\subsection{Math}

Mathematical expressions are input between \verb|\(...\)|.  Display
math is begun and ended with \verb|\[...\]|.  For details of math
input, see \cite{LT}.
%(Math input is also described in \cite{FG} and \cite{TB}, in
%increasing order of complexity of expressions handled.)











\section{Starting a \DProc\/ article}

The first step in preparing an article is to create a file.
The first two lines in this file should be
\begin{verbatim}
\documentstyle{deproc}
\begin{document}
\end{verbatim}
This will cause the formatting definitions to be loaded when the file
is input to \LaTeX.

Next, enter the ``top matter''.  This consists of such
things as the title of the article, the author(s) and their addresses,
and the abstract.

\subsection{Title and authors}

For an article with a short title and one author, the input looks like this:
\begin{verbatim}
\title{A One-Line Title}
\author{Author Name\\
    Author's Organization\\
    City, State}
\end{verbatim}
The double backslashes \verb"\\" indicate line breaks.  This technique is
also used to break up long titles:
\begin{verbatim}
\title{Here We Have a Particularly
    Long Title\\That Can't Possibly
    Fit on a Single Line}
\end{verbatim}
This will be set (in a boldface font slightly larger than text size) as\strut
\begin{center}
        \bf Here We Have a Particularly Long Title\\
        That Can't Possibly Fit on a Single Line
\end{center}
Notice that the way the lines are broken in the input file is not how
they appear in the output\Dash only \verb"\\" matters to \TeX.
Actually, \TeX\ will break long titles into lines short enough to fit
on the page, but a multi-line title usually makes more sense to the reader
if the author decides where the line breaks should occur.

For multiple authors, the same \verb"\author" tag is used with
\verb"\and" or \verb"\And":
\begin{verbatim}
\author{First Author
  \and
    Second Author\\
    Common Organization\\
    City, State}
\end{verbatim}
or
\begin{verbatim}
\author{First Author\\
    First Organization\\
    City, State
  \And
    Second Author\\
    Second Organization\\
    City, State}
\end{verbatim}
and so forth, which will appear thus in the output:\strut
\begin{center}
        {\bf First Author\enspace{\rm and}\enspace Second Author}\\
        Common Organization\\
        City, State
\end{center}
\par\noindent
or\strut
\begin{center}
        \interlinepenalty=10000
        {\bf First Author}\\
        First Organization\\
        City, State\\
        \hspace{10pt}\\
        {\bf Second Author}\\
        Second Organization\\
        City, State\endgraf
\end{center}
%\noindent
Authors' names (the first line, and the first line after \verb"\And") are
printed in boldface; if an author name is to appear on any other line,
begin that line with \verb"\bf" (the \TeX\ instruction for boldface type).

The title and author of the present paper look like this in the file:
\begin{verbatim}
\title{Typesetting Articles for the DECUS
    Proceedings with \LaTeX}
\author{Barbara N. Beeton\\
    \AMS\\
    Providence, Rhode Island}
\end{verbatim}
One item to look at here is \verb"\AMS", which becomes \AMS\ in the output.
This is an example of a ``local definition'', something that is not likely
to be useful to anyone else, but can save the author a lot of time correcting
typing errors.  Local definitions that are used throughout an article are
best input right after specifying the document style:
\begin{verbatim}
\documentstyle{deproc}
\newcommand{\AMS}{American
           Mathematical Society}
...
\begin{document}
\end{verbatim}

\subsection{Abstract}

The abstract is the final part of the top matter.
\begin{verbatim}
\begin{abstract}
This is a short summary of what
the article is about.
\end{abstract}
\end{verbatim}
\par\noindent
The heading ``{\bf Abstract}'' is provided automatically; don't input it.
The abstract may contain more than one paragraph.  Paragraphs are
separated by a blank line or by \verb"\par", as usual.

The top matter is now complete.  The body of the article follows.
\begin{itemize} \parskip=0pt \itemsep=0pt
\item[] \verb"\maketitle"
\item[] (\it{Text of footnotes to the top matter is given here\/})
\item[] {}
\item[] \verb"This is the first sentence of article text."
\item[] \verb"..."
\item[] \verb"\end{document}"
\end{itemize}











\section{The body of the article}

An article can start out with text or with a heading.
Three levels of headings are provided by \DP:
\begin{verbatim}
\section{Section heading}
\subsection{Subsection heading}
\subsubsection{Subsubsection heading}
\end{verbatim}
These produce headings (with extra space above and below, not shown here)
in the following styles:
$$\vbox{\advance\baselineskip by 2pt
\centerline{\bf Section heading}
\leftline{\bf Subsection heading}
\leftline{\it Subsubsection heading}}
$$

The first paragraph following a heading will not be indented in the
default style.  Other paragraphs will be indented a standard amount.
To suppress indentation on a single paragraph, precede it by \verb"\noindent".

\subsection{Footnotes}

A footnote consists of two parts, the mark and the text.  These
are usually entered as a unit\footnote{Like this.}:
\begin{verbatim}
... as a unit\footnote{Like this.}:
\end{verbatim}
This is equivalent to the two statements\footnotemark
\footnotetext{Or this.}
\begin{verbatim}
... two statements\footnotemark
\footnotetext{Or this.}
\end{verbatim}
The two-statement form must be used for footnotes in the title or
abstract and in ``boxed'' environments
 (which will not be explained here; see
\cite{LT} for details).  In such cases, the \verb"\footnotetext" should
be specified as soon as possible after completion of the special
environment.

Footnotes are automatically numbered sequentially starting with 1.
Numbers may also be given explicitly, between \verb"[...]"
following the \verb"\footnote..." command.  In most contexts, this is
optional, but for footnotes in abstracts or in ``boxed'' environments,
the number {\sl must\/} be given for the \verb"\footnotetext"; the
first footnote in this article was produced by the following:
\begin{verbatim}
\maketitle
\footnotetext[1]{\TeX\ is a trademark
       of the \AMS.}
\end{verbatim}

Footnote numbers can be reset if necessary by
\begin{itemize}
\item[] \verb"\resetcounter{footnote}{"{\it integer\/}\verb"}"
\end{itemize}


\subsection{Quotations}

Short quotations, of less than a paragraph, are set with
\begin{verbatim}
\begin{quote}
If you can't fix it, ... {\em Button}
\end{quote}
\end{verbatim}
and look like this:
\begin{quote}
If you can't fix it, call it a feature. \ {\em Button}
\end{quote}

\noindent
For longer quotations, use
\begin{verbatim}
\begin{quotation}
...
\end{quotation}
\end{verbatim}
in a similar manner, separating paragraphs with blank lines as usual.


\subsection{Lists}

Itemized and enumerated lists occur in many \DProc\/ articles.
\LaTeX\ provides automatic counters and up to four levels of nesting.
Here is a short example of a two-level itemized list.
\begin{verbatim}
\begin{itemize}
  \item first item
  \item second item
 \begin{itemize}
   \item new level
   \item one more
 \end{itemize}
  \item back a level
\end{itemize}
\end{verbatim}
\par\noindent
Here's what the output looks like, after padding out the text a bit to show
how longer items look.
\begin{itemize}
\item The first item in this list isn't particularly interesting,
        but it has to be long enough to make two lines.
\item The second item isn't either.
\begin{itemize}
\item Even going to a new level doesn't add very much
        excitement to this exercise.
\item We'll do one more at this level.
\end{itemize}
\item Then we'll go back a level to finish things off.
\end{itemize}
%\noindent
If \verb"{enumerate}" is specified instead of \verb"{itemize}", the
items will be numbered\Dash 1, 2,\dots\ at the first level,
a, b,\dots\ at the second level.  If the default labels aren't what
you want, an overriding label may be specified, for example,
\mbox{\verb"[--]"} (used in Figure~\ref{figtype}).  Each item
comprises one paragraph; an unlabeled paragraph can be produced by
specifying an empty label.  Extra space above and below a list is
provided automatically.

\begin{figure}[t,b]
\hrule
\begin{itemize}
\item[--] Small figures which can be set in place, i.e., in the same
        relative position where they occur in the input file
\item[--] One-column figures to be set at the top or bottom of the
        first available column
\item[--] Double-column figures to be set at the top or bottom of the
        first available page
\item[--] Full-page figures
\end{itemize}
\vspace{-.5\baselineskip}
\caption[figtype]{Possible figure formats}
\vspace{.5\baselineskip}
\hrule
\label{figtype}
\end{figure}

\subsection{Figures}

Figures come in the sizes, shapes and page locations listed in
Figure \ref{figtype}.  Not all these formats are supported yet by \DP\@.
In particular, two-column figures cannot be placed at the
bottom of text pages.

\subsubsection{One-column figures}

To get a single-column figure, enter
\begin{itemize} \parskip=0pt \itemsep=0pt
\item[] \verb"\begin{figure}["{\it loc\/}\verb"]"
\item[] {\it content of figure}
\item[] \verb"\caption{Caption text}"
\item[] \verb"\label{"{\it label\/}\verb"}"
\item[] \verb"\end{figure}"
\end{itemize}
\par\noindent
{\it loc\/} is the location where the figure is to be placed, specified
by one to four letters (in the order in which you find the possible
locations suitable), as follows:
\begin{itemize}
\item[\verb"h"] here, at the position in the text where the figure is input.
\item[\verb"t"] at the top of a text page.
\item[\verb"b"] at the bottom of a text page.
\item[\verb"p"] on a page of ``floats''.
\end{itemize}
\par\noindent
The default {\it loc\/} is \verb"tbp"; note that figures will not be
placed in-line unless \verb"h" is indicated explicitly.  For
additional details, see \cite{LT}.

Figures are numbered automatically.  The optional {\it label\/} is
provided to permit references in the text to the figure:
\begin{itemize}
\item[] \verb"... shown in Figure~\ref{"{\it label\/}\verb"}"
\end{itemize}

To reserve space for figures which will be prepared separately,
use the command \verb"\vspace{"{\it dimension\/}\verb"}".
Some space is automatically skipped above and below a figure, and also
between the figure and the caption, so the dimension given with
\verb"\vspace" should be precisely the height of the item to be pasted in.


\subsubsection{Two-column figures}

Double-column figures can be placed either at the top of a text
page or on a separate page of ``floats''.  The command syntax is the same
as for one-column figures, except that the ``star'' notation
\verb"\begin{figure*}...\end{figure*}" is used.


\subsubsection{Full-page figures}

Full-page figures are a special case of two-column figures,
specified by \verb"\begin{figure*}[p]...\end{figure*}".  A full-page
figure will be placed on the first available page.  If space is to
be reserved for insertion of a separately-prepared figure, use the
\verb"\vspace" command with a suitably large dimension; \verb"8.5in"
should be sufficient.


\subsection{Tables}

\LaTeX\ contains powerful table-formatting capabilities.  However, since
their use is specialized and the rules somewhat complex, specifics
are not presented here; see \cite{LT} for details.


\subsection{Verbatim}

Verbatim items are printed in so-called ``typewriter'' style, using \TeX's
\verb"\tt" font.  In-text verbatim items are preceded by the command
\verb"\verb" and enclosed within a pair of identical characters which
do not occur within the verbatim string (except a space, a letter,
or \verb"*"), for example, vertical bars \verb"|...|" or ditto marks
\verb|"..."|.  Blocks of verbatim code are delimited by
\begin{itemize} \parskip=0pt \itemsep=0pt
\item[] \verb"\begin{verbatim}"
\item[] \verb"..."
\item[] \verb"\end{verbatim}"
\end{itemize}
\verb"\begin{verbatim}" and \verb"\end{verbatim}" should be on lines
by themselves.  Within verbatim
mode, \CR s are obeyed as line breaks, not spaces.  An input line that is too
long for the current column width will be broken at a space if possible,
and the remainder of the line hanging indented on the next output line;
since this may change the meaning of the verbatim passage, such passages
should be checked with special care in the output.  Overlong lines also
frequently result in overfull \verb"\hbox"es, which are listed in the
transcript file.  To mark overfull \verb"\hbox"es clearly on the printed
output (with black boxes: \rule{5pt}{\ht\strutbox}\thinspace), specify
\begin{verbatim}
\documentstyle [draft]{deproc}
\end{verbatim}
at the beginning of your input file.
%{\vrule height\ht\strutbox depth\dp\strutbox width 5pt }\thinspace.
The \verb"draft" option can easily be removed when you are ready to
prepare the final copy.

A passage occurring between \verb"\begin{verbatim}..."\allowhbreak
\verb"\end{verbatim}"
is treated as a unit by \LaTeX\Dash if it is too long for the vertical
space available, it either will be carried over as a unit to the next
column or page, or will result in an overfull \verb"\vbox", which will
be noted only in the transcript of the \LaTeX\ run.  In such a case,
the best remedy is to break the passage in two, by inserting another
\verb"\end{verbatim} \begin{verbatim}".

Verbatim mode is suitable for program listings, indicating keyboarding
instructions, file names, and similar uses.

\subsection{References, bibliography}

References in text to items in the bibliography are input as
\begin{itemize}
\item[] \verb"\cite{"{\it label\/}\verb"}"
\item[] \verb"\cite["{\it text\/}\verb"]{"{\it label\/}\verb"}"
\end{itemize}
where the label is the same as that specified for the item in the bibliography.
For a label ``ABC'', the reference will be rendered \pseudocite{ABC}.
See \cite[pp.~73, 189]{LT} for further details.

References may also be made to figures, tables, or anything else for which
you have established a label:
\begin{itemize}
\item[] \verb"\ref{"{\it label\/}\verb"}"
\item[] \verb"\pageref{"{\it label\/}\verb"}"
\end{itemize}
\verb"\pageref" is not immediately useful\Dash the page numbers
generated by \TeX\ will not be the same as those assigned when the
collection is put together in the editorial office.  This feature would
become useful if it ever becomes possible to submit articles electronically,
as \LaTeX\ input files.

Before you start to input the reference list, some housekeeping is
required\Dash you must decide what you want the list to look like.
This is what the input looks like for one of the items in the reference
list at the end of this article:
\begin{verbatim}
\bibitem[TB]{TB} Knuth, Donald E., \TB,
    Addison-Wesley and \AMS, 1984.
\end{verbatim}
(\verb"\TB" and \verb"\AMS" are among the local definitions for this article.)
Default output looks like this:
\smallskip
{\begin{pseudobibliography}{ACP}
\bibitem[TB]{TB} Knuth, Donald E., \TB, Addison-Wesley and \AMS, 1984.\endgraf
\end{pseudobibliography}
}
\smallskip

Labels may also be omitted, or numbered sequentially.  Begin the
reference list with this command:
\begin{itemize}
\item[] \verb"\begin{thebibliography}{"{\it widest label\/}\verb"}"
\end{itemize}
This will generate the {\bf References} section heading, and establish
the item indentation, using the width of the specified label.

If you do not wish to use labels, substitute \verb"\omit" for the widest
label.  (The \verb"{...}" are still required in the context of
\verb"\bibitem".)
\begin{verbatim}
\bibitem{} Knuth, Donald E., ...
\end{verbatim}
will result in
\par\nobreak\smallskip
{\begin{pseudobibliography}{\omit}
\bibitem{} Knuth, Donald E., \TB, Addison-Wesley and \AMS, 1984.\endgraf
\end{pseudobibliography}
}
\smallskip
\noindent Note that \verb"\cite" cannot be used if references are unlabeled.

If your labels are numeric, no labels need be entered\Dash items will
automatically be numbered sequentially.  However, the widest label
must be specified at the beginning of \verb"thebibliography" environment.
The input
\begin{verbatim}
\begin{thebibliography}{99}
\bibitem{TB} Knuth, Donald E., ...
\end{verbatim}
will now look like this:
\par\nobreak\smallskip
{\begin{pseudobibliography}{99}
\bibitem[2]{TB} Knuth, Donald E., \TB, Addison-Wesley and \AMS, 1984.\endgraf
\end{pseudobibliography}
}











\section{Caveats}

\DP\ and this article were created on a DECSYSTEM-20 at the
\AMS, running \LaTeX\ version~2.08 under
\TeX\ version~1.3.  The AMS installation is standard in all ways.

With one exception, none of the changes to the \TeX\ program since
version~1.0 should have any noticeable effect on an article produced
with \DP.  The exception is large, complex tables\Dash
tables incorporating many boxes and rules require large amounts of
\TeX\ memory.  Memory management was radically changed in version~1.3
to make more memory available to the user without actually changing
the physical memory allotment.  (Otherwise, if you run out of memory,
the most likely cause is an input error.)

Although thorough testing has been attempted, no one outside the AMS has
tried to use \DP\ yet, so bugs are sure to be found.  In fact,
the version of \DP\ first placed in the Program Library should best
be considered a beta test version.  If you find a bug, please
communicate it to the author, accompanied by an example which
demonstrates the bug as simply as possible.  Suggestions for
improvements are also welcome.  Send everything to\break
{\obeylines %
        \indent Barbara Beeton
        \AMS
        \POBox 6248
        Providence, RI 02940
}











%\Bibliography{ACP}
\begin{thebibliography}{ACP}

\bibitem[DP]{DP} Beeton, Barbara, Typesetting articles for the DECUS
        Proceedings with \TeX, {\sl Proceedings of the Digital Equipment
        Computer Users Society}, USA Spring 1985, 349--356.

\bibitem[ACP]{ACP} Knuth, Donald E., {\sl The Art of Computer Programming},
        Addison-Wesley, Vol.~2, second edition, 1981.

\bibitem[TB]{TB} Knuth, Donald E., \TB, Addison-Wesley and \AMS, 1984.

\bibitem[LT]{LT} Lamport, Leslie, {\sl L\kern-.36em\raise.3ex\hbox{\small A}%
        \kern-.15em\TeX, A document preparation system}, Addison-Wesley,
        1985.

\bibitem[TD]{TD} Southall, Richard, First principles of typographic design
        for document production, {\sl\tub\/} Vol.~5 (1984), No.~2, 79--90;
        Corrigenda, Vol.~6 (1985), No.~1, p.~6.

\bibitem[TUB]{TUB} {\sl\tub, the Newsletter of the \TUG}, \TUG, \careof\AMS,
        \POBox 9506, Providence, RI, 02940.

\end{thebibliography}

\end{document}

