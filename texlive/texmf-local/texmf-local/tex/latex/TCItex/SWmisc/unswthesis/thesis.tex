%%% [ This file is provided for the purpose of being an example of using
%%%   the "unswthesis" class.  As such, any information relating to the
%%%   thesis itself has been removed. ]

%%%%%%%%%%%%%%%%%%%%%%%%%%%%%%%%%%%%%%%%%%%%%%%%%%%%%%%%%%%%%%%%%%%%%%%%%
%                                                                       %
%                      University of New South Wales                    %
%               School of Computer Science and Engineering              %
%                                 Thesis                                %
%                                                                       %
%                        (C) John Zaitseff, 1995.                       %
%                                                                       %
%%%%%%%%%%%%%%%%%%%%%%%%%%%%%%%%%%%%%%%%%%%%%%%%%%%%%%%%%%%%%%%%%%%%%%%%%

% Author:       John Zaitseff <J.Zaitseff@unsw.edu.au>
% Date:         October, 1995. [ Stripped 28/8/96 ]
% Format:       LaTeX-2e

%     This document contains the Undergraduate Thesis for John Zaitseff,
% completed at the University of New South Wales in Session 2, 1995.

%@@@ introduces places where something needs to be done.

%%%%%%%%%%%%%%%%%%%%%%%%%%%%%%%%%%%%%%%%%%%%%%%%%%%%%%%%%%%%%%%%%%%%%%%%%
\documentclass[final]{unswthesis}
\usepackage{mythesis}

%%%%%%%%%%%%%%%%%%%%%%%%%%%%%%%%%%%%%%%%%%%%%%%%%%%%%%%%%%%%%%%%%%%%%%%%%
\begin{document}
\frontmatter

\maketitle
\tableofcontents
\listoffigures
\listoftables

\mainmatter

%%%%%%%%%%%%%%%%%%%%%%%%%%%%%%%%%%%%%%%%%%%%%%%%%%%%%%%%%%%%%%%%%%%%%%%%%
\chapter{Introduction}\label{ch:intro}

One of the fastest growing areas of computing today is the area of
portable, often hand-held, devices.  These highly-integrated computers
are used in increasingly many areas, especially as Personal Digital
Assistants, including hand-held data loggers, bar code scanners and meter
readers.

This thesis describes the design of one such portable system, based on
the \Elan microprocessor.  This system features the use of the current
state-of-the-art technology, including high-density surface-mounted
components, low battery power consumption, and directly integrated
support for the new \PCMCIA standard.

%%%%%%%%%%%%%%%%%%%%%%%%%%%
\section*{Acknowledgements}

%%%%%%%%%%%%%%%%%%%%%%%%%%%%%%%%%%%%%%%%%%%%%%%%%%%%%%%%%%%%%%%%%%%%%%%%%
\chapter{Applications}\label{ch:apps}

\section{Biomedical applications}

Such a set-up would look something like Figure~\ref{fg:aging}.
%
\begin{ourfigure}
%\includegraphics{homemon.ps}
includegraphics...
\caption{Remote monitoring of health status in the home}\label{fg:aging}
\end{ourfigure}

The r\^ole of the \EPC, the system this thesis describes, in such
biomedical systems would be to actually take the place of the hand-held
device taking data readings, or as part of the instrumentation modules
placed around the house.  The requirements of the \EPC in such systems
are described in the next chapter.

%%%%%%%%%%%%%%%%%%%%%%%%%%%%%%%%%%%%%%%%%%%%%%%%%%%%%%%%%%%%%%%%%%%%%%%%%
\chapter{Alternatives}\label{ch:alt}

In Chapter~\ref{ch:apps}, we saw a number of applications in which a
portable computer may be used.

...

The \Elan CPU chooses the configuration mode at reset time, on the rising
edge of \pnob{RESIN}, by sampling the state of three pins: \pnob{DTR},
\pnob{RTS} and \pn{SOUT}.  These pins are usually used for serial port
output, except for their special function at reset.  To select one of the
modes, we place \res{10}{k} pull-up or pull-down resistors on these pins,
as per Table~\ref{tb:elmode}.  In this table, a ``1'' indicates pull-up,
``0'' indicates pull-down, and ``X'' indicates that no resistor is
required.
%
\begin{ourtable}
\begin{tabular}{|c|c|c|l|}
\hline
\multicolumn{3}{|c|}{Pin state at reset}  &  \\
\cline{1-3}
\rule[1.3ex]{0mm}{1.3ex}\pnob{DTR} & \pnob{RTS} & \pn{SOUT} &
	\multicolumn{1}{|c|}{\raisebox{1.5ex}[0cm][0cm]{Mode Selected}} \\
\hline
0  &  0  &  X  &  Internal CGA \\
1  &  0  &  0  &  Local bus, 1 \by clock \\
1  &  0  &  1  &  Local bus, 2 \by clock \\
X  &  1  &  X  &  Maximum ISA \\
\hline
\end{tabular}\par
\begin{tabular}{c@{\,}c@{\,}l}
0 &=& pull-down resistor (to \pn{GND}) \\
1 &=& pull-up resistor (to \pn{VCC5}) \\
X &=& no resistor \\
\end{tabular}
\caption{Selection of the \Elan operating mode}\label{tb:elmode}
\end{ourtable}

%%%%%%%%%%%%%%%%%%%%%%%%%%%%%%%%%%%%%%%%%%%%%%%%%%%%%%
\subsection{Level 1 ISA Bus Support}\label{sc:el:isa1}

Table~\ref{tb:elisa1} shows the meaning of the
pins used in this mode.
%
\begin{ourtable}
\renewcommand{\thefootnote}{\textit{\alph{footnote}}}
\begin{tabular}{|l|c|l|}
\hline
Pin Name		& Type%
%
\footnote{I~=~Input, O~=~Output, I/O~=~Bidirectional.  These are with
	  respect to the \Elan device itself.}
%
				& Function \\
\hline
\pn{AEN}		& O	& DMA address enable (DMA cycle)\\
\pn{TC}			& O	& DMA terminal count\\
\pn{SYSCLK}		& O	& System clock%
				\footnote{ISA bus timing is \emph{not}
				derived from this signal: it is meant to
				be used for the keyboard controller
				only.}\\
\pn{PIRQ1}		& I	& Programmable interrupt request~1\\
\pn{DRQ2}		& I	& DMA channel 2 request\\
\pn{X1OUT} / \pn{BAUDOUT}
			& O	& Video clock or serial port clock\\

\pnob{MCS16}		& I	& Memory device is \bit{16}%
\footnote{Note that \pnob{MCS16}, \pnob{IOCS16}, \pnob{SBHE} and
	      \pn{IRQ14} are \emph{not} available in dual-scan LCD mode.} \\
\pnob{IOCS16}		& I	& I/O device is \bit{16}\footnotemark[3]\\
\pnob{SBHE}		& O	& Byte high enable\footnotemark[3]\\
\pn{IRQ14}		& I	& Interrupt request~14\footnotemark[3]\\
\hline
\end{tabular}
\renewcommand{\thefootnote}{\textrm{\arabic{footnote}}}
\caption{ISA bus support, common subset}\label{tb:elisa1}
\end{ourtable}

Before we continue, a few points to note about the notation:
%
\begin{itemize}
    \item  A pin name like \pn{PIN} indicates either an
	   \emph{active-high} pin (i.e., where the pin is asserted when
	   it is at a \High level, generally either \threevolt or
	   \fivevolt), or a \emph{rising-edge} pin (i.e., where the pin
	   is asserted on the rising edge, from \Low to \High, of a
	   pulse).
    \item  A name like \pnob{PIN} is the opposite: it indicates an
	   \emph{active-low} pin, or one that is asserted on the
	   \emph{falling-edge} (\High to \Low) of a signal.
    \item  A name like \pnm{PIN}{3}{0} is \emph{shorthand} for pins
	   individually named \pn{PIN0}, \pn{PIN1}, \pn{PIN2} and
	   \pn{PIN3}.
    \item  All pins are with reference to the microprocessor, i.e.,
	   ``Output'' means output \emph{from} the processor to external
	   peripherals.
\end{itemize}

The following is a description of the pins listed in
Table~\ref{tb:elisa1}.  Much of this information is from pages 39 to~49
of the \book{Data Book}~\cite{ci:data}, although some critical
information is not listed in \emph{any} part of the supplied
documentation:
%
\begin{pindescr}
    \iopin {\pnm{SA}{23}{0}}%
	   {System address bus}%
	   {output, active high}

		The system address bus outputs the physical memory or I/O
		address.  It is used by all external devices, except
		system DRAM\@.  In local bus mode, this represents the CPU
		local address, except that \pn{SA0} is not used.  Note
		that pins \pnm{SA}{23}{13} are multiplexed with the
		memory bus; see section~\ref{sc:el:mem} for more details.

    \iopin {\pnm{D}{15}{0}}%
	   {System data bus}%
	   {bidirectional, active high}

		The system data bus inputs data during memory or I/O read
		cycles, and outputs data during memory and I/O write
		cycles.  In local bus mode, as well as in DRAM read/write
		mode, this bus is used to represent the CPU data bus.
\end{pindescr}

...

These modes are summarised in
Table~\ref{tb:pmmodes}; see also pages~9--15 of the \book{Data
Book}~\cite{ci:data}.
%
\begin{ourtable}
\begin{tabular}{|p{0.15\linewidth}|p{0.6\linewidth}|}
\hline
Mode		&Description\\
\hline\hline
Full speed	&All clocks are at fastest speed and all peripherals are
		powered up.\\\hline
Low speed	&CPU clock is reduced to a lower speed; all other clocks
		are at full speed.\\\hline
Doze		&CPU, system and DMA clocks, as well as the high-speed
		phase-locked loop, are stopped.\\\hline
Sleep		&Additional clocks and peripherals are stopped (depending
		on the programmed settings), as is the serial port
		controller (UART).\\\hline
Suspend		&A special BIOS routine is invoked to save the system
		state, then virtually all of the system is powered down.
		The phase-locked loops are turned off.\\\hline
Off		&A powered-down mode in which \pn{PGP2} and \pn{PGP3} are
		set to a predefined state.  Memory refresh is still
		active.  No activity can cause the processor to leave
		this state, except for a power-on reset.\\
\hline
\end{tabular}
\caption{Power Management Unit operating modes}\label{tb:pmmodes}
\end{ourtable}

%%%%%%%%%%%%%%%%%%%%%%%%%%
\subsection{Other Remarks}

One of the additional features of the internal video controller is that
it may be programmed to appear in the I/O location for a CGA controller
(\addr{3D4}--\addr{3DA}) or in the I/O location for a Hercules Graphics
Adapter (\addr{3B4}--\addr{3BF}).

%%%%%%%%%%%%%%%%%%%%%%%%%%%%%%%%%%%%%%%%%%%%%%%%%%%%%%%%%%%%%%%%%%%%%%%%%
\chapter{Conclusions}\label{ch:concl}

In conclusion, this project has been \emph{very} challenging, but very
interesting as well.  Although we were not able to proceed with the
implementation, we are more than reasonably satisfied that the design is
sound and thorough.  If the project is continued in the next year, the
\EPC will eventually emerge as a symbol of today's modern computing.


%%%%%%%%%%%%%%%%%%%%%%%%%%%%%%%%%%%%%%%%%%%%%%%%%%%%%%%%%%%%%%%%%%%%%%%%%
\begin{thebibliography}{199}
\ssp

%%%%%%%%%%%%%%%%%%%%%
\bibitem{ci:poqetpad}

``\book{Development of a Range of \PCMCIA Type III Instrumentation Modules for
Clinical Measurement, based on the PoqetPad Plus Pen-based Computer}'',
Branko Celler, Biomedical Systems Laboratory, University of NSW,
Australia, 23rd June 1993.

%%%%%%%%%%%%%%%%%
\bibitem{ci:data}

``\book{\Elan \AmSC Microprocessor Data Sheet}'',
Mobile Computing Products Division of Advanced Micro Devices, Inc.

%%%%%%%%%%%%%%%%
\bibitem{ci:prm}

``\book{\Elan \AmSC Microprocessor Programmers' Reference Manual}'',
Mobile Computing Products Division of Advanced Micro Devices, Inc.

%%%%%%%%%%%%%%%%%%%%
\bibitem{ci:intlist}

``\book{PC Interrupt List}''
Ralf Brown, Release~47, August 1995.
Available as\linebreak\texttt{ftp://ftp.cs.cmu.edu/afs/cs.cmu.edu/user/ralf/pub/inter47*.zip}


%%%%%%%%%%%%%%%%%%%
\bibitem{ci:pcspec}

``\book{\PCMCIA Standards}'',
``\book{PC Card Standard 2.1}'',
``\book{Socket Services Specification 2.1}'',
``\book{Card Services Specification 2.1}'',
``\book{PC Card ATA Specification 1.02}'',
``\book{AIMS Specification 1.01}'',
``\book{Recommended Extensions 1.0}'',
Personal Computer Memory Card International Association,
July 1993.

\end{thebibliography}

%%%%%%%%%%%%%%%%%%%%%%%%%%%%%%%%%%%%%%%%%%%%%%%%%%%%%%%%%%%%%%%%%%%%%%%%%
\appendix
\ssp
\chapter{\Elan Computer Schematics}\label{ch:schem}

The following pages show the final \EPC schematics.  The actual
schematics are in the following order:
%
\begin{enumerate}
    \item System Block Diagram
    \item \Elan Microprocessor
    \item Miscellaneous
    \item System Memory
    \item Display Interface
    \item \PCMCIA Buffers
    \item \PCMCIA Connectors
    \item Parallel Port
    \item Serial Port
    \item Keyboard Connector
    \item Expansion Connector
    \item DC/DC Power
    \item Power Switching
\end{enumerate}
%
% Leave enough space (13 pages) for the schematics.
\clearpage
\addtocounter{page}{13}

%%%%%%%%%%%%%%%%%%%%%%%%%%%%%%%%%%%%%%%%%%%%%%%%%%%%%%%%%%%%%%%%%%%%%%%%%
\chapter{Component Data Sheets}\label{ch:datasheet}

The following pages contain all of the data sheets for the components
used in the \EPC design, except for the \Elan \AmSC itself (about
300~pages, available seperately as~\cite{ci:data, ci:prm}, and simple
components like resistors, capacitors and diodes.

...

\end{document}
