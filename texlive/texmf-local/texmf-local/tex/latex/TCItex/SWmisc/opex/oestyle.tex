%%%%%%%%%%%%%%%%%%%%%%%%%%%%%%%%%%%%%%%%%%%%%%%%%%%%%%%
%                   File: OEstyle.tex                 %
%                   VERSION: 1.08                     %
%                   Date: April 3, 1998               %
% LaTeX template file for use with OSA OPTICS EXPRESS %
% THIS FILE IS SHORTER, WITH FEWER DESCRIPTIONS OF    %
% TYPOGRAPHY (BOLD, ITALICS, POINT SIZES, ETC.)       %
%                                                     %
% This file requires a substyle file under the LaTeX  %
% Article style, opex.sty .                           %
%                   FOR LATEX 2E USE                  %
%         \documentclass[10pt]{article}               %
%         \usepackage{opex}                           %
%                                                     %
%                   FOR LATEX 2.09 USE                %
%         \documentstyle[opex]{article}               %
%                                                     %
%                   FOR REVTeX 3.0 or 3.1 USE         %
%         \documentstyle[osa,opex]{revtex}            %
%                                                     %
% Copyright 1997, The Optical Society of America      %
%%%%%%%%%%%%%%%%%%%%%%%%%%%%%%%%%%%%%%%%%%%%%%%%%%%%%%%
%           FOR COLORED WWW LINKS USE                 %
%     \documentclass[10pt]{article}                   %
%     \usepackage{opex}                               %
%     \usepackage{color}%             (LATEX 2E)      %
%                                                     %
%           FOR COLOR WITH LATEX 2.09 BE              %
%           CERTAIN YOU HAVE color.sty AND USE        %
%     \documentstyle[opex,color]{article}             %
%                                                     %
%           FOR COLOR WITH REVTeX 3.0 or 3.1 BE       %
%           CERTAIN YOU HAVE color.sty AND USE        %
%     \documentstyle[osa,opex,color]{revtex}          %
%                                                     %
% Copyright 1997, The Optical Society of America      %
%%%%%%%%%%%%%%%%%%%%%%%%%%%%%%%%%%%%%%%%%%%%%%%%%%%%%%%
%\documentstyle[opex,color]{article}
%\documentstyle[osa,opex,color]{revtex}
\documentclass[10pt]{article}
\usepackage{opex}
%\usepackage{color}% DELETE THIS LINE IF YOU DO NOT HAVE color.sty
\begin{document}

\title{Instructions for the preparation of a LaTeX or 
REVTeX manuscript for Optics Express}

\author{Frank E. Harris}
\address{Optical Society of America, Washington, DC 20036-1023 USA}
\email{fharri@osa.org}

\begin{abstract}%
Explicit and detailed rules are given 
for preparing a compuscript 
(i.e., manuscript in electronic form) for 
Optics Express.  Specific commands 
are given for all major compuscript elements 
(such as abstract, headings, figures, 
tables, and references) to achieve optimal 
typographic quality.  Electronic submission
of a PostScript file, or of a tarred, gzipped archive 
containing a LaTeX file and PostScript or EPS 
figures, is described. 
\end{abstract}
\ocis{(000.0000) General}

% The commands, \begin{OEReferences} and \end{OEReferences}
% format the References section according to OpEx standard
% style, showing the title ``References''.
%
% The commands, \begin{OERefLinks} and \end{OERefLinks}
% format the References section according to OpEx standard
% style, if the references also include URLs or other
% unreviewed links.  In this case the title of the section
% is ``References and links''.
%

\begin{OERefLinks}
\item  P. J. Harshman,  T. K. Gustafson, and P. Kelley,
 ``Strong field theory of low loss optical switching and
  wave mixing in a semiconductor quantum well,''  \opex 
  {\bf 3}, 166-168  (1999).
  
\item  C. van Trigt, ``Visual system-response
 functions and estimating  reflectance,'' \josaa
 {\bf 14}, 741-755 (1997).
 
% DELETE THE FOLLOWING REFERENCE IF YOU DO NOT HAVE
% color.sty ON YOUR TeX SYSTEM.
 
\item  N. Gallagher, ``\underline{Chopper},'' 
OpticsNet, 4/23/97, %{\color{blue}
\underline{http://www.osa.org/video/qtime/chopper.mov}%}.

%\end{OEReferences}
\end{OERefLinks}

% The commands \color{blue} and \underline{} may be used 
% to indicate WWW links in the text.  For those who do not
% have the file color.sty on their systems, \underline{}
% alone may be used. color.sty may be downloaded from the
% CTAN archives at ftp://ftp.dante.de , ftp://ftp.SHSU.edu 
% , or ftp://ftp.cdrom.com .
%
% The set of style files released under the name, "hyperref,"
% also available from the CTAN ftp sites, may
% be used to create WWW links that will carry through to
% the PDF file generated from a .tex file.  This method of
% establishing links may be used to create links to online
% databases, cited articles, or software that cannot be
% served from OSA.  Hyperref is not part of the  
% standard LaTeX 2e package at this time.

\section{ Introduction}
Adherence to the style produced by the file 
opex.sty is essential for efficient 
review and publication of submissions. 
Optics Express takes advantage of 
some of the unique opportunities the electronic 
medium offers to allow authors to include 
3-D scenes and QuickTime sequences 
as illustrations within their papers.  
Instructions for submitting 3-D and video are 
included in Section 3. To take advantage 
of this new medium of publishing, compuscripts 
will be treated as camera-ready copy.  
{\it Authors should print out this style 
guide for use as a checklist for format.} 

Note that the Optics Express rules as to style 
and layout are followed. This has been designed to 
be used in conjunction with the Instructions to 
Authors for Optics Express that can be found at 
(http://epubs.osa.org/opticsexpress/submission/). 

\section{Software}
\subsection{TeX and LaTeX}
The current version of TeX software is 3.14159.  Virtually
all versions of TeX now available as freeware, shareware,
or commercially, are acceptable.  Opex.sty has been tested
extensively with LaTeX 2e and LaTeX 2.09.  Authors who
place WWW links in references are encouraged to 
underline the link. 

Authors who wish to submit LaTeX files are encouraged to
create a tarred, gzipped archive of their LaTeX file and
all figures, in PostScript or EPS format.  Tar and gzip
archiving and compression software is available as 
shareware or freeware for Windows and Mac OS, and is part
of the Unix operating system.  Authors who submit a 
LaTeX file with no figures may submit an ASCII text file
without compression or archiving.

\subsection{Postscript drivers}
Authors who submit a PostScript file to {\it Optics Express} 
should use an appropriate PostScript printer 
driver to generate the compuscript file for submission.  
Almost all versions of dvips will generate 
proper output. The BaKoMa PostScript versions of the 
Computer Modern fonts must be used for submissions
of PostScript files.


\subsection{REVTeX and LaTeX support}
REVTeX and LaTeX commands for title, author, 
address, e-mail, and PACS are supported as aliases
of the OpEx commands listed below.  In limited testing,
REVTeX and LaTeX article style compuscripts have been
found to process with none to two LaTeX errors per
manuscript.  With very little effort a compuscript 
prepared in LaTeX or REVTeX can be converted to OpEx
style.

The OpEx style file, opex.sty, has been tested
under LaTeX 2e (both in native and 2.09 compatability 
mode) and under LaTeX 2.09.  Document styles 
\begin{verbatim}
  \documentclass[10pt]{article}
  \usepackage{opex}
\end{verbatim}
for LaTeX 2e and 
\begin{verbatim}
  \documentstyle[opex]{article}
\end{verbatim}
for LaTeX 2.09 have produced more consistent results in testing, than
\begin{verbatim}
  \documentstyle[osa,opex]{revtex},
\end{verbatim}
the appropriate style option for REVTeX.

Authors who want to place blue text indicating 
WWW links in an article may use the style file 
color.sty, with one of the following: 
\begin{verbatim}
  \documentclass[10pt]{article}% LaTeX 2e
  \usepackage{opex}
  \usepackage{color}
\end{verbatim}
\begin{verbatim}
  \documentstyle[opex,color]{article}% LaTeX 2.09
\end{verbatim}
\begin{verbatim}
  \documentstyle[osa,opex,color]{revtex}% REVTeX.
\end{verbatim}

%\pagebreak

\subsection{Title}
Authors should
place the title within the curly brackets of the 
\verb+\title{}+ command to achieve the correct 
format.

\subsection{Author name}
Author names should appear 
as used for conventional publication, 
with first and middle names or initials followed by 
surname.  Every effort should be made to keep 
author names consistent  from one paper to the next 
as they appear within OSA publications.  Authors should
place their names within the curly brackets of the 
\verb+\author{}+ command to achieve the correct 
format.

\subsection{Author affiliation}
Authors should place their affiliation 
within the curly brackets of the 
\verb+\address{}+ command to achieve the correct 
format. If several 
authors have the same affiliation, one listing of 
the affiliation should be used, preceded by the full 
list of those authors on the line above.  

\subsection{E-Mail address}
Enter the e-mail address of the corresponding author 
directly below the corresponding author�s affiliation.  
Authors should place  
e-mail addresses within the curly brackets of the 
\verb+\email{}+ command to achieve the correct 
format.

Multiple use of the \verb+\author{}+, 
\verb+\address{}+, and \verb+\email{}+ commands 
are permitted and produce clear association of authors
with their institutions, as shown below. 

\author{A. Gatti and L.A. Lugiato}
\address{Instituto Nazional di Fisica per la Materia, 
Dipartmento Di Fisica, Via  Celoria
 16, 20133 Milano, Italy}

\email{lugiato@mi.infi.it}
\author{S. K. Nash-Stevenson}
\address{NASA-Marshall Space Flight Center, EB22,
Huntsville, Alabama 35726}

\author{G-L. Oppo and R. Martin}
\address{Department of Physics, University, of Strathclyde, 
Rottenrow 107, Glasgow G4 ONG, Scotland}

\email{gianluca@phys.strath.ac.uk}


\subsection{Abstract}
Authors should place their abstract between 
\verb+\begin{abstract}+ and \verb+\end{abstract}+ 
commands to achieve the correct format.  The 
\verb+\begin{abstract}+ command will add the word 
{\bf Abstract:} to the front of the abstract.
The abstract should be limited to approximately 100 words. 
It should be an explicit summary of the paper that states 
the problem, the methods used, and the major results and 
conclusions.   A copyright statement will be added 
automatically, after the abstract.


\subsection{OCIS subject classification}
Optics Classification and Indexing Scheme (OCIS) subject classifications 
should be included after the abstract, on a 
separate line that begins with the words {\bf OCIS codes:} 
in bold.  Authors should
place their OCIS codes within the curly brackets of the 
\verb+\ocis{}+ command to achieve the correct 
format.  The OCIS codes should be placed after the abstract,
\verb+\end{abstract}+ command.  The REVTeX
command \verb+\pacs{}+ may be used in place of \verb+\ocis{}+.

\subsection{Main text}
The first line of the first paragraph of a section or 
subsection should start flush left.  The first line of 
subsequent paragraphs within the section or subsection 
should be indented.

\subsection{Equations}
Equations should be centered.  For long equations, the 
right side of the equation should be broken into 
approximately equal parts and aligned to the right of 
the equal sign.  The equation number should appear only 
at the righthand margin of the last line of the equation.

\begin{equation}
H = \frac{1}{2m}(p_x^2 + p_y^2) + \frac{1}{2} M{\Omega}^2
     (x^2 + y^2) + \omega (xp_y - yp_x)
\end{equation}
All equations should be numbered in the order in which 
they appear and should be referenced  from within the 
main text as Eq. (1), the Hamiltonian, Eq. (2), intensity 
distribution, etc. 

%\newpage


\begin{eqnarray}
I_{(x,y,z,t)} & = & \frac{|E_{(x,y,z,t)}|^2}{2} \nonumber \\
     & = & \frac{1}{2}|A(t - \tau_+ z)|^2 \psi_+^2(x,y)
         + \frac{1}{2}|A(t - \tau_- z)|^2 \psi_-^2(x,y) \nonumber \\
     & &  + A(t - \tau_+ z) A(t - \tau_- z)
         \times \cos\left( \pi \frac{z}{L_c}\right) \psi_+(x,y) \psi_-(x,y)
\end{eqnarray}
where

\begin{eqnarray}
I_{(z, \tau)} & = & \frac{1}{2} \left[ \left| 
             A \left(\tau - \frac{\delta \tau}{2}z \right) \right|^2 
           + \frac{1}{2} \left| A \left( \tau + \frac{\delta \tau}{2}z \right) \right|^2 
           + 2A \left( \tau - \frac{\delta \tau}{2}z \right) 
             A \left( \tau + \frac{\delta \tau}{2}z \right) \right. \nonumber \\
     & & \left. \times \cos \left( \pi \frac{z}{L_c} \right) \right] 
             \int _{-\infty}^{+\infty}\int _{-\infty}^{+\infty}
             {\psi_1}^2(x,y){\rm d}x{\rm d}y  \nonumber \\
     & & + \frac{1}{2} \left[ \left| A \left( 
           \tau - \frac{\delta \tau}{2}z \right) \right| ^2 
           + \frac{1}{2} \left| A \left( \tau + \frac{\delta \tau}{2}z \right) \right| ^2 
           - 2A \left( \tau - \frac{\delta \tau}{2}z \right) 
              A \left( \tau + \frac{\delta \tau}{2}z \right) \right. \nonumber \\
     & & \left.  \times \cos \left( \pi \frac{z}{L_c} \right) \right] 
           \int _{-\infty}^{+\infty}\int _{-\infty}^{+\infty}
           {\psi_2}^2(x,y){\rm d}x{\rm d}y  
\end{eqnarray}

In-line math of simple fractions should use parentheses 
when necessary to avoid ambiguity, for example, to 
distinguish between $1/(n-1)$ and $1/n-1$.  Exceptions to 
this are the proper fractions such as $\frac{1}{2}$  
which are better left in this form.  
Summations and integrals that appear 
within text such as  $\frac{1}{2}{\Sigma }  _{n=1}^{n=\infty}
(n^2 - 2n)^{-1}$ 
should have limits placed to the 
right of the symbol to reduce white space.

\subsection{References and links}
References should appear at the top of the article, 
below the abstract, in the order in which they are 
referenced in the body text. 
Authors who use REVTeX may use the standard REVTeX 
commands for references and citations in place of the 
commands listed below. Bibtex may not be used, except as 
described in the OSA bibtex instruction accompanying 
REVTeX 3.1.  The {\bf References}
section must be delimited by horizontal rules above 
and below the section, which may be added manually
if the {\it Optics Express} commands are not used.

 The commands, \verb+\begin{OEReferences}+ 
 and \verb+\end{OEReferences}+ format the {\bf References} 
 section according to OpEx standard style, showing the
 title {\bf References}.  Use the \verb+\item+ command
 to start each reference.  If the references include 
 URLs or other WWW (World Wide Web) links, 
the commands, then \verb+\begin{OERefLinks}+ and 
 \verb+\end{OERefLinks}+ should be used.  If 
 \verb+\begin{OERefLinks}+ is used, the title of 
 the section will be {\bf References and links}.

Optics Express uses numerical notations in brackets 
for bibliographic citations, within square brackets[1] 
or as numerical superscripts.$^2$  When using the 
\verb+\ref{}+ command, labels must be placed in the references:
\begin{verbatim}
   \item\label{paper2}  C. van Trigt, ``Visual system-response 
   functions and estimating 
   reflectance,'' \josaa {\bf 14}, 741-755 (1997).
\end{verbatim}
Then the citation command should be of the form: 
\verb+[\ref{paper2}]+.

Optics Express follows the following citation style: For 
journal articles, authors are listed first, followed by 
the journal�s title abbreviation, the volume number in 
bold, inclusive page numbers and the year in parenthesis.  
This may be followed by a URL if one is available.
The number of a grant or contract should be omitted 
unless its inclusion is required by the agency supporting 
the research.

\vskip1pc

\noindent {\footnotesize
\begin{tabular}{lll}
 \hskip0.2in &2. &C. van Trigt, \josaa {\bf 14},
  741-755 (1997).
\end{tabular}}

\vskip1pc

For monographs in books, authors are listed first, followed 
by article�s full title in quotes, the word �in,� followed 
by the book title in italics, the editors of the book in 
parenthesis, the publisher, city, year.  

\vskip1pc

\noindent {\footnotesize
\begin{tabular}{lll} 
 &3. &David F. Edwards, �Silicon (Si)� in 
  {\it Handbook of optical constants of solids}, E.D. Palik, 
   ed.  \\ & &(Academic, Orlando, Fla. 1985) \\
\end{tabular}}

\vskip1pc

For citation of a book as a whole or book chapter, authors or 
editors are listed first, followed by title in italics, and 
publisher, city, and year in parenthesis.  Chapter number may 
be added if applicable.

\vskip1pc

\noindent{\footnotesize 
\begin{tabular}{lll} 
 &4. &F. Ladouceur and J. D. Love,
   {\it Silica-based buried channel waveguides and devices}
    \\ & & (Chapman \& Hall, 1995), Chap.8. \\
\end{tabular}}

\vskip1pc

WWW links may be represented by a line in the references 
section of the compuscript. The URL of the referred item 
should be underlined.  WWW links should list the author, 
title (substitute file name, if needed), and the full URL 
(universal resource locator).

\vskip1pc

\noindent{\footnotesize
\begin{tabular}{lll} 
 &5. &QuickTime movie viewer plugins for web
 browsers, {\it Welcome to QuickTime}, \\ & & (Apple Computer, 
 1997),    \underline{http://quicktime.apple.com} \\
\end{tabular}}

\vskip1pc

To assist authors with journal abbreviations in references, 
standard abbreviations for 31 commonly cited journals 
have been included as macros within opex.sty.  These abbreviations 
are shown in Table 1.



\section{Tables, Figures, 3-D, and video}

\subsection{Figures}
Figures should be included directly in the document.  
All photographs must be in digital form and placed 
appropriately in the electronic document.  All  
illustrations must be numbered consecutively (i.e., 
not by section), with Arabic numbers. The size of a 
figure should be commensurate with the amount and 
value of the information conveyed by the figure.
Enclose figures within the \verb+\begin{figure}+
and \verb+\end{figure}+ commands.

All illustrations should be centered, except for 
small figures no wider than 6.6 cm (2.6 in), which may 
be placed side by side. Use the \verb+\centerline{}+
command. Place figures as close as possible to 
where they are mentioned in the text. No 
part of a figure should go beyond the typing area.
Authors should use PostScript or EPS figures. Place 
the file name of the figure within the 
curly brackets of the 
\verb+\epsfbox{}+ command. To resize figures, the 
\verb+\epsfxsize=+ command may be used, before 
the \verb+\epsfbox{}+ command, e.g.,
\verb+\epsfxsize=5in \epsfbox{OpExF1.EPS}+.  Figures 
created in the picture environment may be used if 
the submission system processes the figures with 
no errors. In general if a LaTeX file with PostScript 
(PS) or (EPS) figures is sent, the figures must also 
be included in the tarred, gzipped archive. 

Short figure captions should be 
\verb+\footnotesize+ and centered beneath the figure.  
The abbreviation Fig. for figure should appear first 
followed by the figure number and a period.  Longer 
figure captions should be indented 1.27cm (0.5in) 
on both margins.  The \verb+\caption{}+ command
sets type size, margins, and numbers figures
automatically.   The text of captions
should follow normal full-sentence structure.  



\subsection{Tables}
Tables should be centered and numbered consecutively. 
Short table headings should be centered above the 
table.  Longer headings should have margins that 
match the table width.  Tables should use horizontal 
lines to delimit the top and bottom of the table and 
column headings.  Detailed explanations or table 
footnotes should be typed directly beneath the table.  
Position tables as close as possible to where they 
are mentioned in the main text.  

\begin{center}{\footnotesize Table 1. Standard abbreviations
 for 31 commonly cited journals}
\end{center}
\begin{table}
\begin{center}
\begin{tabular}{lp{1.7in}|lp{1.7in}}\hline 
Macro & Abbreviation & Macro & Abbreviation \\ \hline
\verb+\ao+ & Appl.\  Opt.\  & \verb+\nat+ & Nature (London)   \\
\verb+\ap+ & Appl.\  Phys.\  & \verb+\oc+ & Opt.\ Commun.\   \\
\verb+\apl+ & Appl.\ Phys.\ Lett.\
  & \verb+\opex+ & Opt.\ Express   \\
\verb+\apj+ & Astrophys.\ J.\  & \verb+\ol+ & Opt.\ Lett.\   \\
\verb+\bell+ & Bell Syst.\ Tech.\ J.\
  & \verb+\pl+ & Phys.\ Lett.\   \\
\verb+\jqe+ & IEEE J.\ Quantum Electron.\
  & \verb+\pra+ & Phys.\ Rev.\ A   \\
\verb+\assp+ & IEEE Trans.\ Acoust.\ Speech Signal Process.\
  & \verb+\prb+ & Phys.\ Rev.\ B   \\
\verb+\aprop+ & IEEE Trans.\  Antennas Propag.\ 
  & \verb+\prc+ & Phys.\ Rev.\ C   \\
\verb+\mtt+ & IEEE Trans.\ Microwave Theory Tech.\
  & \verb+\prd+ & Phys.\ Rev.\ D   \\
\verb+\iovs+ & Invest.\ Ophthalmol.\ Vis.\ Sci.\
  & \verb+\pre+ & Phys.\ Rev.\ E   \\
\verb+\jcp+ & J.\ Chem.\ Phys.\
  & \verb+\prl+ & Phys.\ Rev.\ Lett.\   \\
\verb+\jmo+ & J.\ Mod.\ Opt.\
  & \verb+\rmp+ & Rev.\ Mod.\ Phys.\   \\
\verb+\josa+ & J.\ Opt.\ Soc.\ Am.\  & 
  \verb+\pspie+ & Proc.\ Soc.\ Photo-Opt.\ Instrum.\ Eng.\   \\
\verb+\josaa+ & J.\ Opt.\ Soc.\ Am.\ A 
  & \verb+\sjqe+ & Sov.\ J.\ Quantum Electron.\   \\
\verb+\josab+ & J.\ Opt.\ Soc.\ Am.\ B
  & \verb+\vr+ & Vision Res.\   \\
\verb+\jpp+ & J.\ Phys.\ (Paris)  & &  \\ \hline
\end{tabular}
\end{center}
\end{table}



\subsection{3-D and video}
Figures may be linked to video or 3-D multimedia.
The figure should be a representative frame of the 
animation.  QuickTime video and 3-D animation 
may be uploaded to OSA by following the instructions 
on the {\it Optics Express} submissions pages.
Include any special instructions concerning links 
in the cover message accompanying submission.
 

%  \pagebreak \epsfbox{filename.ps}
\noindent
\begin{figure}
\centerline{\epsfxsize=5in \epsfbox{OpExF1.EPS}}
\caption{Sample EPS figure showing Bose-Einstein 
condensate data. Use the caption command for longer 
figure captions. REVTeX or LaTeX will automatically 
number the figures.}
%\begin{center} 
%{\footnotesize Fig. 1. Short figure caption.}
%\end{center}
\end{figure}

Authors may submit the resulting PS 
file. Other authors who use LaTeX or REVTeX must make
a tarred, gzipped archive of their .TeX file and 
PS or EPS figures. 


\section{Conclusion}
Conforming to the {\it Optics Express} style is of 
critical importance to the speedy publication of a 
compuscript. After proofreading the compuscript, 
authors who have the capability to make PS files 
with BaKoMa fonts, may print  
PostScript to a file using BaKoMa fonts and submit the 
PostScript file to the journal.  Authors who prefer to
submit LaTeX files must use tar and gzip compression
to archive the manuscript and figures.  If there is 
video or other multimedia, the associated 
files must be uploaded separately.


Enter the requested information into the OpEx online 
submission system at 
http://epubs.osa.org/opticsexpress/submission, and 
upload the file.  For instructions, please see the 
{\it Optics Express} help files.
\end{document}
