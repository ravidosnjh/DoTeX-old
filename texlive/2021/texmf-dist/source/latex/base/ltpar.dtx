% \iffalse meta-comment
%
% Copyright (C) 1993-2021
% The LaTeX Project and any individual authors listed elsewhere
% in this file.
%
% This file is part of the LaTeX base system.
% -------------------------------------------
%
% It may be distributed and/or modified under the
% conditions of the LaTeX Project Public License, either version 1.3c
% of this license or (at your option) any later version.
% The latest version of this license is in
%    https://www.latex-project.org/lppl.txt
% and version 1.3c or later is part of all distributions of LaTeX
% version 2008 or later.
%
% This file has the LPPL maintenance status "maintained".
%
% The list of all files belonging to the LaTeX base distribution is
% given in the file `manifest.txt'. See also `legal.txt' for additional
% information.
%
% The list of derived (unpacked) files belonging to the distribution
% and covered by LPPL is defined by the unpacking scripts (with
% extension .ins) which are part of the distribution.
%
% \fi
%
% \iffalse
%%% From File: ltpar.dtx
%
%<*driver>
% \fi
\ProvidesFile{ltpar.dtx}
             [2021/04/13 v1.1c LaTeX Kernel (paragraphs)]
% \iffalse
\documentclass{ltxdoc}
\GetFileInfo{ltpar.dtx}
\title{\filename}
\date{\filedate}
 \author{%
  Johannes Braams\and
  David Carlisle\and
  Alan Jeffrey\and
  Leslie Lamport\and
  Frank Mittelbach\and
  Tobias Oetiker \thanks{Tobi did the documentation update}\and
  Chris Rowley\and
  Rainer Sch\"opf}
\begin{document}
 \MaintainedByLaTeXTeam{latex}
 \maketitle
 \DocInput{\filename}
\end{document}
%</driver>
% \fi
%
%
%
% \changes{v1.1a}{1994/05/16}{(ASAJ) Split from ltinit.dtx.}
% \changes{v1.1b}{1995/04/29}{(TO) Comments clean-up.}
%
% \section{Paragraphs}
%
% This section of the kernel declares the commands used to set
% |\par| and |\everypar| when ever their function needs to be
% changed for a long time.
%
% This file here describes the interfaces that have been in the kernel
% forever, used to implement the scenarios described below. They
% remain valid but are now augmented in the next file
% (\texttt{ltpara.dtx}) to add hooks to paragraphs. At some point we
% will consolidate the two files further.
%
%
% There are two situations in which |\par| may be changed:
%
% \begin{itemize}
% \item Long-term changes, in which the new value is to remain in effect
%     until the current environment is left.  The environments that
%     change |\par| in this way are the following:
%     \begin{itemize}
%         \item All list environments (itemize, quote, etc.)
%         \item Environments that turn |\par| into a noop:
%              tabbing, array and tabular.
%     \end{itemize}
%  \item  Temporary changes, in which |\par| is restored to its previous
%  value the next time it is executed. The following are all such uses.
%      \begin{itemize}
%        \item |\end| when preceded by |\@endparenv|, which is called by
%                 |\endtrivlist|
%        \item  The mechanism for avoiding page breaks and getting the
%           spacing right after section heads.
%       \end{itemize}
% \end{itemize}
%
%
% \StopEventually{}
%
% \subsection{Implementation}
%
%
% \DescribeMacro{\@setpar}
% To permit the proper interaction of these two situations, long-term
% changes are made by the |\@setpar{|\meta{VAL}|}| command.
% It's function is:
% \begin{quote}
%    To set |\par|. It |\def|'s |\par| and |\@par| to \meta{VAL}.
% \end{quote}
%
% \DescribeMacro{\@restorepar}
% Short-term changes are made by the usual |\def\par| commands.
% The original values are restored after a short-term change
% by the |\@restorepar| commands.
%
% \DescribeMacro{\@@par}
% |\@@par| always is defined to be the original \TeX{} |\par|.
%
% \DescribeMacro{\everypar}
% |\everypar| is changed only for the short term.  Whenever |\everypar|
% is set non-null, it should restore itself to null when executed.
%
% The following commands change |\everypar| in this way:
% \begin{itemize}
%    \item |\item|
%    \item |\end| when preceded by |\@endparenv|, which is called by
%                 |endtrivlist|
%    \item |\minipage|
% \end{itemize}
%
% When dealing with |\par| and |\everypar| remember the following two
% warnings:
% \begin{enumerate}
%  \item Commands that make short-term changes to |\par| and |\everypar|
%     must take account of the possibility that the new commands and the
%     ones that do the restoration may be executed inside a group.  In
%     particular, |\everypar| is executed inside a group whenever a new
%     paragraph  begins with a left brace.  The |\everypar| command
%     that restores its  definition should be local to the current
%     group (in case the command
%     is inside a minipage used inside someplace where |\everypar| has
%     been redefined).  Thus, if |\everypar| is redefined to do an
%     |\everypar{}| it could take several executions of |\everypar|
%     before the restoration ``holds''.  This usually causes no problem.
%     However, to prevent the extra executions from doing harm,
%     use a global switch to keep anything harmful in the new
%     |\everypar| from being done twice.
%   \item Commands that change |\everypar| should remember that
%   |\everypar| might be supposed to set the following switches false:
%      \begin{itemize}
%        \item |@nobreak|
%        \item |@minipage|
%      \end{itemize}
%      they should do the setting if necessary.
%  \end{enumerate}
%    \begin{macrocode}
%<*2ekernel>
\message{par,}
%    \end{macrocode}
%
%
% \begin{macro}{\@setpar}
% \begin{macro}{\@par}
%    Initiate a long-term change to |\par|.
%    \begin{macrocode}
\def\@setpar#1{\def\par{#1}\def\@par{#1}}
%    \end{macrocode}
%
%    The default definition of |\@par| will ensure that if
%    |\@restorepar| defines |\par| to execute |\@par| it will redefine
%    itself to the primitive |\@@par| after one iteration.
%    \begin{macrocode}
\def\@par{\let\par\@@par\par}
%    \end{macrocode}
% \end{macro}
% \end{macro}
%
% \begin{macro}{\@restorepar}
%    Restore from a short-term change to |\par|.
%    \begin{macrocode}
\def\@restorepar{\def\par{\@par}}
%</2ekernel>
%    \end{macrocode}
% \end{macro}
%
% \Finale
\endinput

