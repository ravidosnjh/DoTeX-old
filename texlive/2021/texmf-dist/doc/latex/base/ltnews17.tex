% \iffalse meta-comment
%
% Copyright (C) 1993-2021
% The LaTeX Project and any individual authors listed elsewhere
% in this file.
%
% This file is part of the LaTeX base system.
% -------------------------------------------
%
% It may be distributed and/or modified under the
% conditions of the LaTeX Project Public License, either version 1.3c
% of this license or (at your option) any later version.
% The latest version of this license is in
%    http://www.latex-project.org/lppl.txt
% and version 1.3c or later is part of all distributions of LaTeX
% version 2008 or later.
%
% This file has the LPPL maintenance status "maintained".
%
% The list of all files belonging to the LaTeX base distribution is
% given in the file `manifest.txt'. See also `legal.txt' for additional
% information.
%
% The list of derived (unpacked) files belonging to the distribution
% and covered by LPPL is defined by the unpacking scripts (with
% extension .ins) which are part of the distribution.
%
% \fi
% Filename: ltnews17.tex
%
% This is issue 17 of LaTeX News.

\documentclass
%    [lw35fonts]    % uncomment this line to get Palatino
     {ltnews}[2004/02/28]

% \usepackage[T1]{fontenc}


\publicationmonth{December}
\publicationyear{2005}
\publicationissue{17}

\providecommand\pkg[1]{\texttt{#1}}
\providecommand\cls[1]{\texttt{#1}}
\providecommand\option[1]{\texttt{#1}}
\providecommand\env[1]{\texttt{#1}}
\providecommand\file[1]{\texttt{#1}}

\begin{document}

\maketitle

% \raisefirstsection

\section{Project licence news}

The \LaTeX{} Project Public License has been updated slightly so that
it is now version 1.3c.  In the warranty section the
phrase ``unless required by applicable law'' has been reinstated,
having got lost at some point.  Also,
it now contains three clarifications: of the difference between
``maintained'' and ``author-maintained''; of the
term ``Base Interpreter''; and when clause 6b and 6d shall not apply.

Following requests, we now also provide the text of the licence as a
\LaTeX{} document (in the file \texttt{lppl.tex}). This file can be
processed either as a stand-alone document or it can be included
(without any modification) into another \LaTeX{} document, e.g., as an
appendix, using \verb|\input| or \verb|\include|.


\section{New guide on font encodings}

Way back in 1995 work was started on a guide to document the
officially allocated \LaTeX{} font encoding names. However, for one
reason or another this guide (named \textit{\LaTeX{} font encodings})
was, until now, not added to the distribution.  It describes the major
7-bit and 8-bit font encodings used in the \LaTeX{} world and explains
the restrictions required of conforming text font encodings.  It also
lists all the `encoding specific commands' (the LICR or \LaTeX{}
Internal Character Representation) for characters supported by the
encodings \texttt{OT1} and \texttt{T1}.

When the file \file{encguide.tex} is processed by \LaTeX{}, it will
attempt to typeset an encoding table for each encoding it describes.
For this to be possible, \LaTeX{} must be able to find \texttt{.tfm}
files for a representative example font for each encoding.  If
\LaTeX{} cannot find such a file then a warning is issued and the
corresponding table is omitted.


\section{Robust commands in math}

The font changing commands in text-mode have been robust commands for years,
but the same has not been true for the math versions such as
\verb|\mathbf|.  While the math-mode commands worked correctly in
section heads, they could cause problems in other places such as index
entries.  With this release, these math-mode commands are now robust in
the same way as their text-mode counterparts.

%%
\pagebreak
%%


\section{Updates of required packages}

Several of the packages in the \textsf{tools} bundle have been updated
for this release.

The \pkg{xspace} package has some new features.  One is an
interface for adding and removing the exceptions it knows about and
another is that it works with active characters. These remove problems
of incompatibility with the \pkg{babel} system.

In \textit{\LaTeX\ News~16} we announced that some packages might
begin to take advantage of \eTeX{} extensions on systems where these
are available: and the latest version of \pkg{xspace} does just
that.  Note also that \pkg{fixltx2e} will make use of the
facilities in \eTeX{} whenever these are present (see below).

The \pkg{calc} package has also been given an update with a few
extra commands.
The commands \verb|\maxof| and \verb|\minof|, each with two
brace-delimited arguments, provide the usual numeric $\max$ and $\min$
operations. The commands \verb|\settototalheight| and
\verb|\totalheightof| work like \verb|\settoheight| and
\verb|\heightof|.  There are also some internal improvements to make
\pkg{calc} work with some more primitive \TeX\ constructs, such as
\verb|\ifcase|.

The \pkg{varioref} package has acquired a few more default
strings but there are still a number of languages for
which good strings are still missing.

The \pkg{showkeys} package has also been updated slightly to work
with more recent developments in \pkg{varioref}. Also, it now
provides an easy way to define the look of the printed labels with
the command \verb|\showkeyslabelformat|.

\section{Work on \LaTeX{} fixes}

The package known as \pkg{fixltx2e} has three new additions. A new
command \verb|\textsubscript| has been added as a complement to the
command \verb|\textsuperscript| in the kernel.  Secondly, a new form of
\verb|\DeclareMathSizes| that allows all of its arguments to have a
dimension suffix.  This means you can now use expressions such as
\verb|\DeclareMathSizes{9.5dd}{9.5dd}{7.4dd}{6.6dd}|.

The third new
addition is the robust command \verb|\TextOrMath| which takes two
arguments and executes one of them when typesetting in text or math
mode respectively. This command also takes advantage of \eTeX{}
extensions if available; more specifically,
%%
%%\pagebreak
%%
when the \eTeX{} extensions are available, it does not
destroy kerning between previous letters and the text to be typeset.
The command is also used
internally in \pkg{fixltx2e} to resolve a problem with
\verb|\fnsymbol|.

Also, further work has been done on reimplementing
the command \verb|\addpenalty|, which is used internally in several
places: we hope it is an improvement!

\section{The graphics bundle}

The \textsf{graphics} bundle now supports the
\texttt{dvipdfmx} post-processor and Jonathan Kew's XE\TeX\
program. By support we mean that the graphics packages recognize the
new options \option{xetex} and \option{dvipdfmx} but we do not
distribute the respective driver files.

This leads elegantly to a description of the
new policy concerning such driver files in the \textsf{graphics} bundle.
Most driver files for our graphics packages are maintained by the
developers of the associated post-processor or \TeX\ programs.
The teams developing these packages are working very hard: their rapid
development offers a stark contrast to the current schedule of \LaTeX\
releases.  It is therefore no longer practical
for the \LaTeX\ Team to be responsible for distributing the
latest versions of these driver files.

Therefore the installation files for \pkg{graphics} have been
split: there is now \file{graphics.ins} to install the package files and
\file{graphics-drivers.ins} for the driver files (located in
\file{drivers.dtx}).
There is no need
to install all those provided in the file \file{drivers.dtx}.

Please also note that, as requested by the maintainers of
\textsf{PStricks}, we have removed the package \pkg{pstcol} as
current versions of \textsf{PSTricks} make it obsolete.

%%
\newpage
%%


\section{Future development}

The title of this section is a little misleading as it actually
describes \emph{current} development. In 1998 the \textsf{expl3}
bundle of packages was put on \ctan\ to demonstrate a possible \LaTeX3
programming environment. These packages have been lying dormant for some
time while the \LaTeX\ Project Team were preoccupied by other things
such as developing the experimental packages \pkg{xor},
\pkg{template}, etc., (and also writing that indispensable
and encyclopaedic volume,\\
The \LaTeX{} Companion -- 2nd edition).

In October 2004 work on this code base was resumed with the goal of
some day turning it into a kernel for \LaTeX3.  This work can now
also make full use of the widely accepted \eTeX\ extensions. Currently
two areas are central to this work.

\begin{itemize}
\item Extending the
  kernel code of \LaTeX3.
\item Converting the experimental packages
  such as \pkg{xor}, \pkg{template} to use the new
  syntax internally.
\end{itemize}

Beware! Development of
\textsf{expl3} is happening so fast that the descriptions above might
be out of date when you read this! If you wish to see what's going on
then go to
\url{http://www.latex-project.org/code.html} where you can download
 fully working code (we hope!).

\end{document}
